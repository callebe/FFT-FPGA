Como afirma \citeonline[Pref�cio]{Meyer}, muitos algoritmos de processamento de sinais, como FFT \textit{(Fast Fourier Transform)} e os filtros FIR ou IIR, implementados anteriormente em PDSPs ou em Circuitos Integrados de Aplica��o Especifica ou ASIC \textit{(Application Specific Integrated Circuits)}, agora est�o sendo implementados em FPGAs.

\citeonline{moore}[p.~4] define a FPGA como um dispositivo semicondutor capaz de ser totalmente redefinido ap�s sua fabrica��o, permitindo ao desenvolvedor reconfigurar produtos e fun��es j� implementadas, adaptando o \textit{hardware} a novas fun��es. De forma pr�tica, a FPGA permite uma flexibilidade em um projeto, podendo mudar a forma como ele � implementado, sem a necessidade de se construir um \textit{hardware} novo.

\vspace{5mm}
