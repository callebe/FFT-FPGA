Segundo \citeonline[p.~530]{Lathi}  "Um sinal peri�dico $x(t)$ com per�odo $T_0$ pode
ser descrito como a soma de senoides de frequ�ncia $f_0$ e todas as suas harm�nicas", conforme visto em (\ref{eq:SerieFourierFundamental}). O qual e conhecida como s�rie trigonom�trica de Fourier de um sinal peri�dico $x(t)$, sendo $\omega_0$ a frequ�ncia fundamental.


\begin{equation}
	x(t) ~=~ a_0 + sum _{n=1} ^{\infty} a_n cos (n \omega_0 t) + b_n sen(n \omega_0 t)
	\label{eq:SerieFourierFundamental}
\end{equation}

E poss�vel determinar os coeficientes $a_n$ e $b_n$ de (\ref{eq:SerieFourierFundamental} )pelas equa��es (\ref{eq:SerieFourierFundamentalan}) e (\ref{eq:SerieFourierFundamentalbn}), em que o coeficiente $T_0$ representa o per�odo relativo a $f_0$.

\begin{equation}
	a_n~=~\frac{2}{T_0} \int_{T_0} x(t) cos(n \omega_0 t) dt
	\label{eq:SerieFourierFundamentalan}
\end{equation}

\begin{equation}
	a_n~=~\frac{2}{T_0} \int_{T_0} x(t) cos(n \omega_0 t) dt
	\label{eq:SerieFourierFundamentalbn}
\end{equation}

A express�o da s�rie de Fourier em termos exponenciais $e^{j \omega_0 t}$ e $e^{?j \omega_0 t}$ � facilmente obtida a partir da forma trigonom�trica \cite{Lathi}. A forma exponencial da serie de Fourier � dada por (\ref{eq:SerieFourierExponencial}), em que o coeficiente $C_n$ e an�logo aos coeficientes $a_n$ e $b_n$ da serie trigonom�trica, sendo obtido por (\ref{eq:SerieFourierExponencialCn}.) Nota-se que diferente dos coeficientes $a_n$ e $b_n$, o coeficiente $C_n$ pode representar um valor complexo.

\begin{equation}
	x(t)~=~\sum^{\infty} _{- \infty} C_n e^{j n \omega_0 t}
	\label{eq:SerieFourierExponencial}
\end{equation}

\begin{equation}
	C_n~=~ \frac{1}{T_0} \int_{T_0} x(t) e^{j n \omega_0 t} dt
	\label{eq:SerieFourierExponencialCn}
\end{equation}

\subsection{Espectro Exponencial de Fourier}

Segundo \citeonline[p.~556]{Lathi}, o espectro exponencial de Fourier e tra�ado a partir dos coeficientes $C_n$ e das frequ�ncias $n \omega_0$ da forma exponencial da serie de Fourier. Ent�o e necess�rio expressar o espectro em fun��o da parte real e da parte imaginaria, ou do modulo e do �ngulo. A forma de $C_n$ em modulo � angulo geralmente � mais �til para se expressar o espectro. Logo s�o ta�ados dois gr�ficos para o espectro exponencial de Fourier, um que relaciona $|C_n |$ com $n \omega_0$, e outro que relaciona $\angle C_n$ com $n \omega_0$.

Para \citeonline[p.~533]{Lathi} os dois gr�ficos juntos formam o espectro de frequ�ncia, o qual revela os conte�dos de frequ�ncia do sinal $x(t)$, com suas
amplitudes e fase. Conhecendo-se este espectro n�o s� � poss�vel analisar o sinal
$x(t)$, como tamb�m reconstru�-lo de forma f�cil.
